% --- Style pour les Théorèmes et résultats (Italique) ---
\theoremstyle{plain}
\newtheorem{theorem}{Théorème}[chapter]
\newtheorem{proposition}[theorem]{Proposition}
\newtheorem{lemma}[theorem]{Lemme}
\newtheorem{corollary}[theorem]{Corollaire}
\newtheorem{conjecture}[theorem]{Conjecture}

% --- Style pour les Définitions (Texte droit) ---
\theoremstyle{definition}
\newtheorem{definition}[theorem]{Définition}
\newtheorem{example}[theorem]{Exemple}
\newtheorem{exercise}[theorem]{Exercice}
\newtheorem{problem}[theorem]{Problème}

% --- Style pour les Remarques et Notes ---
\theoremstyle{remark}
\newtheorem{remark}[theorem]{Remarque}
\newtheorem{note}[theorem]{Note}
\newtheorem{notation}[theorem]{Notation}

% --- Configuration de Cleveref pour le Français ---
% Permet de taper \cref{th:mon_theoreme} et d'obtenir "théorème 1.1"
\crefname{theorem}{théorème}{théorèmes}
\crefname{proposition}{proposition}{propositions}
\crefname{definition}{définition}{définitions}
\crefname{lemma}{lemme}{lemmes}
