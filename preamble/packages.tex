% --- Encodage et Langue ---
\usepackage[utf8]{inputenc}
\usepackage[T1]{fontenc}
\usepackage[french]{babel} % Gestion des césures et titres en français

% --- Mathématiques (Indispensables) ---
\usepackage{amsmath, amssymb, amsthm}
\usepackage{mathrsfs}   % Pour les tribus (ex: \mathscr{A}, \mathscr{F})
\usepackage{mathtools}  % Pour \DeclarePairedDelimiter (normes, valeurs absolues)
\usepackage{amsfonts}
\usepackage{bbm}        % Pour la fonction indicatrice \mathbbm{1}

% --- Couleurs et Graphiques ---
\usepackage[dvipsnames]{xcolor} % Pour RoyalBlue et les gris
\usepackage{graphicx}
\usepackage{tikz}       % Pour les schémas de systèmes dynamiques
\usetikzlibrary{cd}     % Pour les diagrammes commutatifs
\usepackage[framemethod=TikZ]{mdframed} % Pour les encadrés (ex. le Disclaimer de titlepage)

% --- Mise en page ---
\usepackage{geometry}
\geometry{
    a4paper,
    margin=3cm,
    headheight=14pt % Évite des warnings avec fancyhdr
}
\usepackage{microtype} % Améliore subtilement l'espacement des lettres (très pro)
\usepackage{emptypage} % Rend les pages vraiment vides entre les chapitres

% --- Liens et Références ---
\usepackage{hyperref}
\hypersetup{
    colorlinks=true,
    linkcolor=RoyalBlue,
    citecolor=PineGreen,
    urlcolor=Magenta
}
\usepackage{cleveref}   % Pour \cref{label} qui écrit par exemple "Chapitre 1" tout seul

% --- Bibliographie (Recommandé : BibLaTeX + Biber) ---
\usepackage{csquotes}
\usepackage[backend=biber, style=alphabetic]{biblatex}
\addbibresource{bibliography/references.bib}
